\section{Conclusion}

Censorship has a direct effect on knowledge accumulation by making censored material less available to scholars. It also discourages writers from engaging in non-compliant work, and hence modifies the allocation of talents across different types of activities.
In this paper, we developed a new method that considers these two channels. Then, we applied it to the Catholic Church's censorship from the Counter-Reformation until the Enlightenment. We investigated whether censorship was responsible for the demise of Italian science and evaluated the relative importance of the direct channel vs. the activity choice channel.

Our analysis had three steps. First, we collected data on members of universities and academies, identifying the scholars whose books were either allowed to be printed and sold, or put in the \textit{Index Librorum Prohibitorum},  i.e. censored. Second, we built a theoretical model of knowledge accumulation through book production and censorship, distinguishing non-compliant knowledge (susceptible to being censored) from compliant knowledge. Third, we estimated the structural parameters of the model using facts collected from the dataset. We used the quantitative model to answer our questions by simulating a counterfactual path of knowledge dynamics characterized by the absence of censorship.

We concluded that censorship reduced by 35\% the average log publication per scholar in Italy from 1470-1549 to 1680-1749. Renaissance Italy has been regarded as the cradle of culture and science. Yet, Italy found itself in a scientific backwater during the seventeenth and eighteenth centuries, being overtaken by non-Catholic countries such as Great Britain and the Netherlands. The sizeable effect that we estimated supports a claim that the Church's censorship was one of the main drivers of Italy's decline.

Half of this drop stems from the induced reallocation of talents towards compliant activities, while the other half arises from the direct effect of censorship on book availability. This result stresses the importance of selection effects when analyzing the impact of censorship on output. The top scholars at the time of the Counter-Reformation were  censored (Bruno, Galilei, Copernicus), and their potential successors might have been published as  compliant poets instead.

Finally, one may wonder whether the Church's censorship also had a role in the \textit{economic} decline of Italy. This is not implausible, given that recent research highlighted the role of upper-tail human capital production for pre-industrial Europe's take-off \cite{squicciarini2015,cantoni2014,mokyr2012,mokyr2016}. Our analysis sets the stage for future research on this topic by directly linking the Church's censorship to upper-tail human capital production.

%Renaissance Italy has been regarded as the cradle of culture and science. Yet,  Italy found itself in a scientific backwater during the seventeenth and eighteenth centuries, being overtaken by North and Western Europe. In this paper, we shed light on this reversal by highlighting the role of the Church's censorship against novel ideas. We found that the role of such censorship was prominent as it was able to reduce by 28 the average log publications per scholar in Italy from 1470-1550 to 1680-1750.
%To study the role of censorship on knowledge growth, we developed a new method that accounts for the endogenous selection of agents into compliant vs. non-compliant ideas. Our method consists of developing a new model linking censorship to knowledge diffusion and occupational choice. In the model, authors draw ideas from the stock of knowledge left by the previous generation, retaining the best one.
%Two crucial mechanisms of our model are: $i)$ printers choose their sector according to the relative quality of compliant and revolutionary ideas, $ii)$
%Decomposing highlights the importance of using our novel method accounting for the endogenous selection into different sectors


