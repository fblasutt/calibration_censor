\section{Introduction}
Italy's primacy in knowledge creation was undisputed in the fifteenth and sixteenth century. However Northern and Western Europe overtook Italy in the following two centuries, a period in which scholars and the knowledge they produced are believed to have played an essential role in the rise of the West \cite{mokyr2016}. The first explanation proposed for such a reversal of fortune is the fight led by the Catholic Church against novel ideas \cite{land99}, such as heliocentrism \cite{gingerich1973copernicus}, infinitesimal calculus \cite{alexander2014infinitesimal}, and atomism \cite{beretta2007}.\footnote{Probably Newton would have had issues developing his particle theory of light in a country averse to atomism.} These novel ideas were at the root of the Scientific Revolution in Europe.

We tackle this issue by focusing on the role of one weapon in the Church's arsenal, namely the power to censor books published by scholars. The list of prohibited books is called the \textit{Index Librorum Prohibitorum}. We ask whether this censorship was key in altering the growth path of the generation of new knowledge in the Italian peninsula.

We answer the question with three contributions. First, we construct a large sample of scholars active in Italy from 1400 to 1750 and we document how the intensity of censorship and the (relative) notability of blacklisted authors changed over time. Second, we use this data to identify the deep parameters of a novel model linking censorship to knowledge diffusion, accounting explicitly for agents' endogenous selection into compliant vs. non-compliant activities. Third, we perform a counterfactual experiment to assess quantitatively the role of censorship in the decline in total publications per scholar in Italy.

In the first part of the paper, we build a database of Italian scholars active in the academies and universities from 1400 to 1750. For each scholar, we identify whether his (or her) work was subject to censorship by the Church. We also measure the ``quality'' of each scholar by his (or her) quantity of written output in today's library catalogs. Using this new database, we document the drop in publications per person over the period 1400-1750. Studying the distribution of the publications per person, we highlight that, in the sixteenth century, the censored authors were of much better quality, on average, than the non-censored authors. Moreover, this difference shrunk over time. The intensity of censorship decreased as well, after it was first introduced in the sixteenth century. This pattern may reflect either a deliberate choice of the best authors to switch from non-compliant to compliant publications, or a change in the Church's policy, or both.

In the second part of the paper, we design a structural model linking censorship to knowledge diffusion and productivity growth over the long-run. The model explicitly includes the two mechanisms described in the first part. In the model, knowledge is codified in books and can be of two types: conformist and non-conformist. Following the literature on endogenous growth and knowledge diffusion \cite{kremer1993population,jones2001industrial,lucas2009,lucas2014,de2017clans}, we assume that authors randomly draw ideas from the stock of knowledge left by the previous generation, retaining the best one. We introduce a novel occupational choice made by printers between printing compliant/conformist books or revolutionary/non-conformist books. Revolutionary books are less likely to be printed if they are of lower quality than compliant books.\footnote{In a robustness exercise, we also consider the possibility that authors and printers self-censor because of the fear of being persecuted under the Inquisition.} We show that, by censoring revolutionary books, the Church can not only reduce the share of people in the revolutionary occupation, but, more importantly, can alter the development path of knowledge drastically. Setting up a (costly) censorship apparatus  reduces the spread of revolutionary ideas and forces the society to converge towards a compliant steady state.

The development and estimation of the structural model constitute a new methodology to measure the effect of censorship on knowledge growth. We account for the effect of censorship on the availability of already written books, and for its repercussions for the sector and the quality of future knowledge. This is done by modeling the endogenous selection of agents into the compliant vs. non-compliant sectors, which depends on past knowledge and censorship. The decision by the Church to introduce censorship is also endogenized. Overall, the structure and estimation of the model allow us to build a counterfactual path of knowledge dynamics characterized by the absence of censorship.

In the third and last part of the paper, we use the facts highlighted in the first part to identify the deep parameters of the structural model. The most important parameter, namely the rate of censorship, is intuitively identified by the share of censored authors. The dynamics of the overall quality of authors identify some key technological parameters.  The productivity of censored and non-censored authors is implied by the share of censored authors and overall quality. Without targeting these moments in particular, the model  matches them well, which gives credence to the  model's mechanisms. The fixed cost necessary to impose censorship is picked to match the timing of the creation of the first Index of forbidden books. Simulations show that imposing a censorship rate of 19\% on the non-conformist books was sufficient to decrease the share of non-conformist authors from 51\% in 1470-1550 to 23\% in 1680-1750. We conclude that censorship reduced by 35\% the average log publication per scholar in Italy. Interestingly, half of this drop stems from the induced reallocation of talents towards compliant activities, while the other half arises from the direct effect of censorship on book availability.\footnote{The effect of censorship is also due to the interaction between $i)$ its direct effect and $ii)$ the induced reallocation of talents. We reported the size of $i)$ and $ii)$ assuming that the effect of the interaction is shared between $i)$ and $ii)$ proportionally, according to their relative ``pure'' effects.} The results are robust to several sensitivity checks, including a model extension that accounts for the imperfect enforcement of censorship in the Italian peninsula \cite{putnam1906}. The parameter that governs imperfect censorship is calibrated such that it matches the causal estimates of censorship enforcement in \citeN{becker2021}.

The effect of censorship on knowledge growth can be contrasted with the impact of adverse macroeconomic shocks that struck the Italian economy over the same period. To model such shocks, we assume that the number of books people can buy is proportional to income per capita. If real GDP per capita had remained constant after 1470 instead of dropping by about 20\% \cite{malanima2011long},  the average log publication per scholar  would have been 9\% higher. The effect of adverse macroeconomic conditions on knowledge production is one fourth of  the effect of censorship.




\textbf{Literature} Our paper relates to three strands of the literature. First, we add to the existing literature that studies the effects of censorship. Motivated by the fact that a large share of the world population is currently subject to censorship,\footnote{According to \citeN{freedom}, only 13\% of the world population enjoys a free press.}  previous research studied how autocratic governments strategically impose censorship \cite{king2013,zhuang2019} and its effectiveness in stopping the spread of non-compliant ideas \cite{roberts2014}. This paper contributes to this literature by proposing a novel method to study censorship, accounting for the endogenous selection of agents into compliant vs. non-compliant knowledge. On the theory side, \citeN{shadmehr2015} propose a model where the ruler can censor media reports to avoid revolts, while citizens might update negatively about a regime when they see no news. \citeN{guriev2020} study the trade-offs between various tools of authoritarian politics such as censorship, propaganda and repression. We contribute to this literature by making endogenous the creation and quality of non-compliant content.

Another strand of the literature explores the way government and religious institutions fought against novel ideas in early modern Spain \cite{vidal2011,drelichman2021}, Europe \cite{anderson2015,galbiati22,cabello2022}, Imperial China \cite{koyama2015}, and the Islamic world \cite{iyigun2015,chaney2016,rubin2017}. Relative to these works, this paper differs by distinguishing the effect of censorship from that of the Inquisition. Censorship affects knowledge production by making some ideas unavailable to future generations, while the Inquisition is the enforcement arm of the Church, responsible for punishing heretics. Censorship can be effective even if heretic authors do not risk their life. This paper is also one of the first works in economics about the effect of Catholic censorship, alongside \citeN{becker2021} and \citeN{comino2021censorship}. \citeN{becker2021} study the effect of censorship on the number of printed books, while \citeN{comino2021censorship} focus on the effect of censorship on publishing firms in Venice. Both unravel a causal effect of censorship on publication levels. Instead of taking books or printers as the unit of observation, we focus on scholars and on the decision to comply with the Church's ideology. Focusing on authors also allows us to weight them by quality, and to study the dynamic effects of censorship via diffusion of knowledge to future generations in a structural growth model.

Second, our paper contributes to the literature on changes and persistence in institutions and development \cite{acemoglu2005institutions,henriques2019comparative,johnson2019persecution}. Closely related to our work, \citeN{benabou2015} focus on the persistence of religiosity in a framework where belief-eroding innovations can be censored, and religious institutions can adapt the doctrine to the new knowledge. Ekelund, Hebert, and Tollison \citeyear{ekelund2002,ekelund2004} study the behavior of the Catholic Church before and after the rise of Protestantism by interpreting the Church as an incumbent monopolistic firm. Our is a dynamic approach to understanding the persistence of the Catholic Church's reach, where decisions to impose censorship depend on the current and future (endogenous) distribution of authors' quality by sector. Our framework allows us to rationalize  both the Church's late reaction to the rise of Protestantism and that several books censored in the sixteenth century could circulate freely in the previous centuries.

Finally, this paper is tied to the literature on the root causes of the relative decline of Italy. The hypotheses regarding the demise of Italy include the excessive control by the guilds \cite{cipolla2004}, the inability of Italy to seize the new, profitable transatlantic trade routes \cite{land99,braudel2018}, war and plagues \cite{alfani13,alfani2013calamities}, and the fight of the Catholic Church against novel ideas \cite{land99,gusdorf1971}. We focus on the latter argument by examining the role of the Catholic Church's censorship on knowledge diffusion. Compared to the literature on knowledge diffusion in the Malthusian epoch \cite{de2017clans}, in which knowledge is embodied into craftsmen, we model a complementary vector of idea transmissions by focusing on codified/written knowledge. We do not seek to make a direct link between censorship and economic growth, even though recent research highlights the importance of upper-tail human capital for pre-industrial Europe's take-off \cite{squicciarini2015,cantoni2014,mokyr2012,mokyr2016}.

The remainder of this paper is organized as follows.  In Section 2, we present the data sources, and we highlight two novel facts about censorship and scholar quality. In Section 3, we develop a model linking censorship to knowledge diffusion.  In Section 4, we estimate the structural model and present its implications for the role of censorship on Italy's accumulation of knowledge. %The conclusion is in Section 5.
