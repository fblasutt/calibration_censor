
\section{Robustness Checks}\label{appendix:c}

This section describes the sensitivity checks whose results are reported in table \ref{table:robust}.


\textbf{Only Italian scholars.} Some scholars might have spent only a period of their time in Italy. Living outside Italy could have allowed them to access forbidden books without consequences. To limit this problem, we estimate the model using a sample of Italian scholars only. Table \ref{table:robust} shows that this sensitivity check's results differ only slightly with respect to baseline results.

\textbf{Only southern/northern Italian scholars.} The model used for the baseline estimation assumes that the rate of censorship that the Church can enforce does not depend on scholars' location in Italy. This assumption is problematic if the actual rate of censorship differed drastically across Italian regions. To understand whether this is the case, we estimate the model separately for Italian scholars born in northern and southern Italy. A scholar is defined as northern Italian if he is born in a city whose latitude is larger than $43.8$, which corresponds to cities northern than Florence. The results reported in table \ref{table:robust} indicate that the effect of censorship on knowledge growth is very similar for northern and southern Italian scholars. The effect is slightly stronger for southern Italians because the rate of censorship there is slightly larger. This result is consistent with the stronger capacity that the Church had in the Papal state.

\textbf{Only $t\leq4$.} In the baseline model, we assume that the Church could enforce censorship until 1750, the end of period $t=5$. In this sensitivity check, we re-estimate the model assuming that the Church can enforce censorship until the end of $t=4$ only, or 1680. In the last period $t=5$, the Church keeps censoring authors, but anyone can read revolutionary books. The Church's ability to enforce censorship likely decreased over time. It is also likely that her ability to censor did not disappear completely. Hence, we think that this robustness provides a lower bound to the effect of censorship on knowledge growth. Despite the conservative assumption, the results in table \ref{table:robust} show that the effect of censorship is still large, even though slightly lower than in the baseline case. This is because once the decline of revolutionary ideas started, its decline becomes unstoppable because of inertia. 


\textbf{No weak links.} In our baseline sample, we included scholars who have a weak link to a university or academy. These include foreign and corresponding members to academies. One example is Leonhard Euler at Accademia Ricovrati. While all these scholars decided to do some work with the institution, they might not have been there physically. Scholars with weak links might be less constrained by the Church's censorship: for example, because they lived elsewhere in Europe. Hence, we propose a sensitivity check where we exclude them from the sample and then re-estimate the model. Table \ref{table:robust} reports the results, that differ only slightly from the baseline estimation. One reason why excluding weak links have a low effect on the results is that they represent less than $2\%$ of the original sample.

\textbf{All publications.} In the baseline sample, we measure the author's quality by the number of publications written \textit{by} them. It is possible to argue that quality is better measured if also publications \textit{about} the author are included. These capture the impact that these authors had on future generations. Table \ref{table:robust} reports the results where quality is measured by considering both publication \textit{by} and \textit{about} the author. The role of knowledge accumulation is very similar to the baseline, which indicates that results are robust to different quality measures.

\textbf{Lenght Wikipedia pages} One problem with our measure of authors' quality is that it may be biased because older works have more editions. To limit this problem, we consider a different measure of author’s quality, based on the number of characters of the author's longest Wikipedia page. Table \ref{table:robust} shows that our results are robust to this different measure of quality. Note that for building this measure of authors' quality we followed \citeN{de2020academic} by assuming that having no Wikipedia page is similar to having one page with a length of 60 characters.

\textbf{Imperfect censorship.} In the model, we assumed that none could access the knowledge embodied in forbidden books. As historians \cite{grendler1975} documented the smuggling of forbidden books, this assumption is not always met. This sensibility check consists in assuming that the Church was able to enforce censorship only in $\chi\%$ of total cases. Hence, even if $m_t\overline{\beta}$ authors have been censored, only  $m_t\overline{\beta}\chi$ are not available to the next generation. One important question is how to set the value of $\chi$. Our strategy is the following: we set $\chi$ the lowest possible, under the constraint that, after the re-estimation of the model given $\chi$, it exists one value of $\psi$ rationalizing the timing of censorship (introduced in $t=3$).\footnote{The procedure to set $\psi$ is the one described in section \ref{section:identification}. In particular, the timing of Censorship by the Church can be identified if $\psi_L$<$\psi_R$.} In this way, we give the best shot to this sensibility check to undermine the baseline results. We find $\chi=0.91$. When $\chi$ is too small, the model's estimation implies that the share of revolutionary ideas would be declining already in $t=1$.\footnote{When censorship is not effective, the decrease in quality can be matched only when this happens naturally, without any intervention by the Church.} When this happens, censorship is set immediately, and the timing of censorship cannot be rationalized. Also, low levels of $\chi$ implying $m_1<m_2$ are at odds with the fact that the share of revolutionary authors is larger in 1470-1540 than in 1400-1470. The results reported in table \ref{table:robust} indicate that imperfect censorship has only a minor effect on the baseline results. In particular, the impact of censorship on knowledge growth stays large and negative.

%\begin{itemize}
%	\item Gingerich found that about two thirds of the copies of \textit{De revolutionibus} in Italy at the time of the decree were "corrected." However, virtually none of the copies outside Italy, among which Spain and France, were touched. \citeN{gingerich2004} analyzes 277 copies of the first edition and 324 of the second of \textit{De revolutionibus}. 

%	\item \citeN{marcus2020} finds that nearly six thousand requests for reading licenses, approximately 10 percent of which were granted to physicians. 

%\end{itemize}

\textbf{Model with self-censorship} In section \ref{section:twof} we claimed that, in some cases, the persecution by the Church was not confined to censorship, but the author lost his life or was imprisoned. Hence, it is reasonable to think that some authors might have decided to self censor to avoid risking their life. Others might have migrated elsewhere in Europe, where the Church could not reach them.\footnote{\citeN{de2020academic} show that an European academic market existed in early modern times.} These two mechanisms can decrease the share of revolutionary authors for a reason other than direct censorship. In subsection \ref{subsection:censor} we proposed to take these mechanisms into account by discounting the quality of revolutionary books by $\gamma$. Here we re-estimate the model enriched by this featured. Parameter $\gamma$ is mostly identify by $\overline{\beta}m_2$, which is too low in the simulations when the baseline model is used. Parameter $\gamma$ helps to make the demise of revolutionary ideas faster, thus allowing for an initial larger level of revolutionary ideas. The estimation implies that $\gamma$=0.95 and $\overline{\beta}$=0.16, which is very close to the baseline. Then, we asses the role of direct censorship by comparing simulations with the estimated $\overline{\beta}$ and setting $\overline{\beta}=0$, where $\gamma$ is always set to its estimated value. If the baseline model was misspecified, the version with self censorship should give a different effect of direct censorship on knowledge growth. This is not the case: table \ref{table:robust} shows that the results differ only slightly from the baseline. To understand the joint role of direct and self censorship, we perform a counterfactual simulations where $\overline{\beta}=0$ and $\gamma=1$. The joint effect reduces knowledge quality by $41\%$. Since the effect of direct censorship was $29\%$, this means that also self-censorship has an effect on knowledge quality, even if including it in the model do not alter the baseline results about the effects of direct censorship.

\textbf{Universities only} In the baseline estimation we consider both University professors and members of academies. In appendix \ref{appendix:data2} we show that while the coverage of University professor is very good, we probably miss many member of academies. Hence, we provide a robustness check where we exclude those scholars who were not professors. Table \ref{table:robust} shows that the result of the baseline and this alternative estimation are very similar: censorship reduced log publications by $26\%$ in the first case and $24\%$ in the second case.
	

