
\FloatBarrier
\section{Bibliographies}\label{appendix:b}

\textbf{John Barclay}  (Pont-à-Mousson 1582 - Roma 1621, censored in 1608) was born to a Scottish-born father. In 1605 John Barclay presented the first part of his {\em Euphormionis Lusinini Satyricon}. This humanist novel is a very original piece of work \cite{correard17}, including a  satirical description of the Jesuit schools (he was raised in a Jesuit school). This book was put in the Index on 13 December 1608 \cite{de2002index}. At the invitation of the Pope himself, he went to Rome in 1616 and resided there until he died in 1621. Moving to Rome was a way to signal that he was a good Catholic. John Barclay was a member of several Italian academies, including the Accademia degli Umoristi and the Accademia dei Lincei.


\textbf{Giordano Bruno}  (Nola 1548 - Roma 1600, censored in 1600) was an Italian friar, a member of the Dominicans. His contributions span from philosophy to mathematics and cosmology. He is best known for being persecuted by the Catholic Church and was later regarded as a martyr for science. The Inquisition found him guilty of heresy for several of his views, among which his positions on cosmology: he theorized an infinite universe and a plurality of worlds. All of his works were entered the Index of forbidden books, and he was burned at stake in Rome's square, the Campo de' Fiori.

\textbf{Bernardino Ciaffoni} (Porto Sant'Elpidio 1615/1620 - Marches 1684, censored in 1701) was a theologian and belonged to the order of the Franciscans. He also used to be a rector of the well-known college San Bonaventura, located in Rome. His \textit{Apologia}, published posthumously, defends the rigorist doctrine and fights the probabilism supported by Jesuits. This piece of work was introduced into the Index because of its 'insulting' claims against Jesuits.


\textbf{Nicolaus Copernicus} (Thorn 1474 - Frauenburg 1543, censored in 1616) was a Prussian mathematician and astronomer. In his book \textit{De revolutionibus orbium coelestium}, he theorized the cosmos as having the Sun at the center of the solar system, where the Earth rotated around it. This theory is a deep contrast to the Ptolemaic model, where the Earth is stationary at the center of the universe. Several other scientists, including Galilei, contributed to his theory by bringing evidence to support it. While his theories were welcomed positively by the Church at first, his \textit{De revolutionibus} was censored in 1616, after that the Church's conservative revolution.

%\textbf{Achille Gagliardi} (Padova 1537 -- Modena 1607, censored in 1703) was a Jesuit theologian and spiritual writer. He taught philosophy at the Roman College, then theology in Padua and Milan. He was a collaborator of the Archbishop of Milan Carlo Borromeo, who asked him to write a handbook of religion, the popular \textit{Catechismo della fede cattolica}. His \textit{Breve compendio} was censored because of his thoughts about the annihilation of the will during mystical states. These ideas are not compatible with free will, which is a cornerstone of catholic theology.

\textbf{Galileo Galilei} (Pisa 1564 - Arcetri 1642, censored in 1634) was an Italian astronomer and physicist. Also Professor in Padova and member of the prestigious Accademia dei Lincei, arguably he was the most notable and influential scientist of his times. He is also known as the father of modern science because of his work on the scientific method. His books were censored because of its support to atomism, heliocentrism, and Copernicanism. The Inquisition condemned him, and he was forced to abjure his thesis and spent the last part of his life under house arrest.

\textbf{Serry Jacobus Hyacinthus} (Toulon 1659 – Padua 1738, censored in 1722) was a theologian and belonged to the order of the Dominicans. Also consultor of the Congregation of the Index, he taught theology at the University of Padua from 1698. His \textit{Historiae}, written under the pseudonym Augustinus Leblanc, deals with the Jesuit-Dominican controversy on grace and was prohibited by the Inquisition.
