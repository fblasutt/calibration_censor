\documentclass[12pt]{article}

\usepackage{amsfonts,amssymb,amsmath}
\usepackage{setspace,achicago,graphicx,url}


\usepackage{xr-hyper} % to allow links to the appendix
\externaldocument[O-]{prohibitorum-appendix}
\externaldocument[M-]{prohibitorum}

\parskip5mm

\usepackage[top=.9in, bottom=.9in, left=.9in , right=.9in,letterpaper]{geometry}


%For accepting graphs
\usepackage{graphicx}



\begin{document}

\hfill September 2022

\textbf{MS2022 – 443-1: “Catholic Censorship and the Demise of Knowledge Production in
Early Modern Italy”, by Fabio Blasutto and David de la Croix}


\textbf{Reply to Editor}

We are very grateful to the Editor for giving us the opportunity to revise our paper.
As detailed in our Replies to the two Referees, we have carefully addressed all their comments. We believe that, by doing so, we have managed to substantially improve and streamline both the empirical analysis and the theory presented in our paper.

We have also dealt with all the specific issues raised by the Editor. Below, we report his three main comments/requests in \emph{italics}, followed by our point-to-point replies.
We hope that this revision will meet your expectations.

\subsubsection*{The three main shortcomings}

\textit{I
would like the paper to try and do more to dismiss competing hypothesis on the “fall in
productivity” (such as demographics due to the plague and/or the catching up of the rest
of Europe).}

To address this point, we have extended the analysis along three new lines. 

First, we now consider the supply side effect of the drop in longevity, which is affected by wars and plagues. It is explained in the paper as follows:
``The above simulation only considers the effects of longevity on the demand side of the market for books. Longevity can also affect the supply side, by reducing the time available to authors to write books. This aspect is absent from the model, but we can get an idea of its size by using the data. In Appendix~\ref{O-app:robust-longevity}, we quantify this channel  in two steps. First, we calculate the marginal effect of living one additional year on the mean, median, and 75th percentile of the log-publications of European scholars. We find a highly significant effect according to which one more year of life increases the log median publication by 0.019.
Second, we adjust the baseline distributional moments by adding the marginal effects above times the deviation of aggregate longevity from its value in Period~1.
That is, we calculate what the scholars' publications would look like if Italy did not experience the drop in longevity. We conclude that the drop in longevity experienced by Italy over the period 1470-1680 led scholars to  publish less, reducing the median log publications by 2\% at most. Hence, the supply side effect of the drop in longevity is there, is highly significant, but quite small.''

Second, we consider the hypothesis according to which the loss of urban population generated by wars and plagues may have caused the decline of Italy: `` We focus on urban population as it is more directly related to knowledge formation than the total population.  We first use the new data set of \citeN{buringh2021population} on European cities. We obtain the numbers presented in Table~\ref{M-tab:pop}. According to that source, the urban population in Italy did not fall during the period considered and stalled during the seventeenth century. This reflects that demographic shocks might have been strong for some specific places, but not that strong at the macroeconomic level. We next compare Buríngh's numbers to those proposed by \citeN{alfani2019plague} for 32 cities from their  database which includes only the cities for which complete information about city population every fifty years from 1500 to 1800 was available. Using this data, we observe a drop in urban population between 1600 and 1650. This drop of 7.7\% is close to the drop in longevity we imputed in the exercise above but is less long lasting, so one cannot expect stronger effects using urban population instead of longevity. Moreover, population recovered and even overtook its previous level by 1750.''


To conclude on the role of demographic shocks, it is likely that they affected knowledge production during the seventeenth century, by reducing longevity and/or the size of the urban population. However, they cannot explain why the quality of authors remained so low in our last period (1680-1750), while both longevity and population had recovered. Instead, censorship removed books from libraries and depressed quality \textit{every year} over most of the period (with diminishing enforcement in the last decades), being therefore able to explain the dramatic cumulative effect on quality we observe in the data.

Third, we provide a second out of sample test of the model, simulating the growth of publications in Great Britain, using the parameters of Italy but without censorship and with lower initial publications per scholar. This exercise is related to the catching up of countries starting from a lower initial level of publications and it reinforces the trust in our approach: ``
As a second out of sample test of our theory, we compare the dynamics of publications in Great Britain with model simulations. Data for Great Britain are in Appendix~\ref{O-app:british}. We simulate the dynamics of publications by decreasing the initial conditions in the state of knowledge ($k^R_1,k^C_1$) used for Italy by $45\%$ to match the median number of log publications in Great Britain in $t=2$. The remaining parameters are set equal to those used for Italy, except for $\overline{\beta}$, which is set to 0 (no censorship). We also recompute the process for $\mu_t$ using GDP per capita. The results are shown in Figure~\ref{M-fig:Sq_uk}.

Simulations of median log publications in Great Britain are relatively close to their data counterpart, despite we did not used them in the estimation procedure. The model shows the ability to match well the British data when the initial conditions are set to a level lower than the Italian case. This result supports the claim that the model predicts correctly the catching up and overtaking of Great Britain, through a mix of convergence forces acting in the dynamics of $m_t$, differences in the  exogenous process of $\mu_t$, and the presence of censorship in Italy.
''




\textit{Second, I would like the paper to provide the readers with a clearer anchor
to the literature, connecting to the most recent sources (see the report by referee 1) and
being explicit about what is novel and what is not (in the theory part).}

On the rise and decline of religious persecution, we added a reference to \citeN{johnson2019persecution}.
We also added references to recent papers studying the effect of the Counter-Reformation at the European level: \citeN{galbiati22}, \citeN{cabello2022},
and also to a paper claiming that institutions are not responsible for the decline of Southern Europe: \citeN{henriques2019comparative}.

We also added new references to address points by Referee 1: \citeN{maifreda2014denari}, \citeN{alfani2019plague}, \citeN{malanima2011long}.

To be more explicit about the contribution, we rewrote the beginning of the theory section, saying: ``In this section, we build on the theory of accumulation and dissemination of knowledge through the combination of ideas (Kortum 1997, Lucas 2009, Lucas and Moll 2014). We add to this theory a new occupational choice, which can be biased by the presence of censorship.''


\textit{Third, I would like
you to substantially shorten the paper, to a maximum of 30 pages (same line spacing)
especially the second part (model + structural estimates). I think a shorter paper will make
a sharper and more successful contribution.}

We have shortened the paper from 39 pages to 30 (those two numbers do not include the bibliography). This was achieved by moving material to the online appendix, including the details of the model with optimal censorship, the discussion of model assumptions, the long robustness analysis, and the Figure with the historical sources. We have also deleted some unessential footnotes and sentences, and moved technical details to the appendix.

\subsubsection*{Key issues to be addressed}

\textit{
(1) Robustness of the empirical analysis. My main concern with the empirical analysis is related to the possibility that
the differential trends between Italy vs Europe are also driven by other macro phenomena,
such as the differential incidence of the Plague (point 2 by referee 1) or
the catching up of the underdeveloped countries (see point 5 by referee 2). I think
the paper will be stronger with an open discussion of the alternative hypothesis.
Point 8 by referee 2 is well taken: it would be interesting to know whether the
results also survive with an alternative choice of the time periods.}



Most of these points have been addressed in answering the first shortcoming above (plague and catching up). Concerning the choice of the time periods, we now provide an additional robustness analysis:  In Appendix~\ref{O-app:robust} we propose a sensitivity check where we consider ten model periods that correspond to 1400-1434, 1435-1469, 1470-1504, 1505-1539, 1540-1574, 1575-1609, 1610-1644, 1645-1679, 1680-1714, 1715-1749.  The results of this sensitivity check (reported in Table \ref{O-table:robust} in the Appendix) differ only slightly with respect to baseline results. The benchmark model implies that censorship reduced by 35\% the average log publications per scholar in Italy, while this robustness implies a decline of 37\%.



\textit{
(2) On the foundation and properties of the chosen outcome variable. Both referees
ask well posed questions on your choice of the outcome variable as the total
number of books. (see point 3 and 9 by referee 2). Today one would probably
look for the number of citations, rather than total publications. I understand that
this is not feasible with historic data, but the question remains about what evidence
suggests that the number of publications is a good proxy of the quality of research.}


To address this point, we provide in a new Appendix  (Appendix~\ref{O-app:cor}) the correlation matrix of various measures of quality (publications by and about, number of library holdings, length of Wikipedia page, number of languages in which a Wikipedia page exists).
We conclude there that ``All the notoriety measures are highly correlated with each other, in particular when we use the rank correlation. In the main analysis, we opted for the variable ``publi by'' because it is the one which is the closest to our theoretical concept of books by an author. But we could have taken another one, all the computations in terms of quantiles would not be much affected.''


Moreover, We have written a new appendix (Appendix~\ref{O-app:robustdecline}) in which we provide the summary statistics using these other measures, to see if the decline of Italy is robust to the choice of the quality/notoriety measure. We were not surprised that it is indeed the case.


\textit{
(3) Shorten and sharpen the model. I find the theory part a bit too slow to get to the
point. It is also not clear what is novel and what is taken from the literature (as in
e.g. Kortum, Lucas, Moll). Overall, I think a shorter theory section (with details
and derivations moved to an appendix) will be better.
Do we really need to wait until section 3.5 to get a time varying policy? if the
assumption is important (to fit some data as claimed) then just start right off with
this model. If not, move this modified problem to an appendix.}

See our answer to the third shortcoming above.

\textit{
(4) Improve anchor in the literature and revise all figures. Both referees note that
the paper should do a better job in providing an anchor with the historic period
under scrutiny and the related literature: see point 1 by referee 2 and the remark
by referee 1 who notes that “the survey of the literature on this topic is not very
updated (the most recent cited work is 1999)”. See also point 8 by referee 2.}


Regarding the first part of this point, see our answer to the second shortcoming above.

We replied to point 8 by referee 2 as follows: 

``The censorship effort conducted by Catholic Church was a reaction to the rise in novel ideas that took place in Europe in the sixteenth century. While censorship was intended for all Europe, we focus on Italy mainly because, for political reasons, enforcement has been easier in the Italian peninsula than elsewhere (Putnam 1906). 

Even if the Church did not directly enforce censorship in Spain, Spanish censorship is among the most iconic examples of a state-sponsored apparatus enforcing religious homogeneity. It would indeed be interesting to apply the model to Spain, but, unfortunately, the number of published scholars in Spain is much lower than in Italy, leading to very large standard errors when measuring publications per person. One advantage of focusing on Italy is to have many large, well-documented, universities and many academies. Spain had no academies. Moreover, among the scholars in the large universities, few are published. See our data for Salamanca here  \url{https://ojs.uclouvain.be/index.php/RETE/article/view/66073}, and Valladolid here
\url{https://ojs.uclouvain.be/index.php/RETE/article/view/59013}. Portugal and Spain are merged in Table 1 to increase the number of observations per period (note also that they formed a political union between 1580 and 1640).

About the list of included countries: the database is built around universities and academies, without having specific countries in mind. Still, at the end of the period, universities cover all of Europe except territories under the Ottoman rule and some small countries where there was no higher education institution.

Since we realized that the full database was not well documented in the paper, we wrote a new Appendix on European data. This is Appendix~\ref{O-appendix:full}, which also features a map of the origins of our scholars in Europe."

\textit{
I also find some of the figures not so easy to interpret. You should ask yourself
whether they all 7 figures are necessary. Please also improve the quality of the
figures and make sure both axes are clearly labelled: figure 7 is a bad example:
time (what units?), vertical axis: what variable?. Is figure 1 essential? I have a
hard time distilling a clear message from figure 2 . The vertical scale changes (it
doubles) moving from the first to the second row of the figure panels; also the
sample is small, so it is hard to tell whether differences are due to sampling or to
some fundamental changes.}


Figures 4 to 6 (previously 6 and 7) now benefit from labels on the vertical axes and on the different curves. Former Figure 1 has been moved to the appendix. About former Figure 2 (now 1), we think it conveys the message that censorship was concentrated at the top of the distribution during the first periods, and more and more spread over the whole distribution as time passes. Note that the difference in log publications between censored and non-censored authors in period 1 (year 1400-1469) is statistically different at the $5\%$ level than the difference in log publications between censored and non-censored authors in period 5 (year 1680-1749). We have adjusted the vertical scales.

\bibliographystyle{achicago}
\bibliography{prohibitorum}



\end{document}



\subsubsection*{The key issues to be addressed}

\textit{(1) Robustness of the empirical analysis.
My main concern with the empirical analysis is related to the possibility that
the differential trends between Italy vs Europe are also driven by other macro phenomena,
such as the differential incidence of the Plague (point 2 by referee 1) or
the catching up of the underdeveloped countries (see point 5 by referee 2). I think
the paper will be stronger with an open discussion of the alternative hypothesis.
Point 8 by referee 2 is well taken: it would be interesting to know whether the
results also survive with an alternative choice of the time periods.}

\textit{(2) On the foundation and properties of the chosen outcome variable. Both referees
ask well posed questions on your choice of the outcome variable as the total
number of books. (see point 3 and 9 by referee 2). Today one would probably
look for the number of citations, rather than total publications. I understand that
this is not feasible with historic data, but the question remains about what evidence
suggests that the number of publications is a good proxy of the quality of research.}


\textit{(3) Shorten and sharpen the model. I find the theory part a bit too slow to get to the
point. It is also not clear what is novel and what is taken from the literature (as in
e.g. Kortum, Lucas, Moll). Overall, I think a shorter theory section (with details
and derivations moved to an appendix) will be better.
Do we really need to wait until section 3.5 to get a time varying policy? if the
assumption is important (to fit some data as claimed) then just start right off with
this model. If not, move this modified problem to an appendix.}

\textit{(4) Improve anchor in the literature . Both referees note that
the paper should do a better job in providing an anchor with the historic period
under scrutiny and the related literature: see point 1 by referee 2 and the remark
by referee 1 who notes that “the survey of the literature on this topic is not very
updated (the most recent cited work is 1999)”. See also point 8 by referee 2.}

\textit{and revise all figures.
I also find some of the figures not so easy to interpret. You should ask yourself
whether they all 7 figures are necessary. Please also improve the quality of the
figures and make sure both axes are clearly labelled: figure 7 is a bad example:
time (what units?), vertical axis: what variable?. Is figure 1 essential? I have a
hard time distilling a clear message from figure 2 . The vertical scale changes (it
doubles) moving from the first to the second row of the figure panels; also the
sample is small, so it is hard to tell whether differences are due to sampling or to
some fundamental changes...}

Figures 4 to 6 (previously 6 and 7) now benefit from labels on the vertical axes and on the different curves. Former Figure 1 has been moved to the appendix. About former Figure 2 (now 1), we think it conveys the message that censorship was concentrated at the top of the distribution during the first periods, and more and more spread over the whole distribution as time passes. We have adjusted the vertical scales.




